%Pair Programming
%Refactoring
%Versionsstyring
%Peer review (Inspection)

\chapter{Tools, Techniques and Practices} % (fold)
\label{cha:tools_techniques_and_practices}
This chapter will describe, and present a rationale for the different tools, techniques and practices, which have been chosen.
Lastly the implementation of these will be presented.
The chosen tools, techniques and practices are: \textbf{Pair Programming}, \textbf{Refactoring}, \textbf{Version control} and \textbf{Peer Review (Inspection)}.

\subsection*{Pair Programming} % (fold)
\label{sub:pair_programming}
\begin{description}
    \item[What?]\hfill\\
    When coding two programmers may work together on the same machine/computer.
The process involves two distinct roles; One programmer writes the code and the other reviews each line of code, as it is written. However the two programmers are encouraged to switch roles once in a while, during the pair programming.

    \item[Why?]\hfill\\ 
    
    
    \item[How?]\hfill\\
    
\end{description}
% subsection pair_programming (end)
\subsection*{Refactoring} % (fold)
\label{sub:refactoring}
\begin{description}
    \item[What?]\hfill\\
    When existing code is restructured and/or changed without changing the way it behaves to the user (externally), it is called refactoring.
    This technique is generally used to gain readability and maintainability in a codebase.

    \item[Why?]\hfill\\ 
    
    
    \item[How?]\hfill\\
        
\end{description}                       
% subsection refactoring (end)
% chapter tools_techniques_and_practices (end)