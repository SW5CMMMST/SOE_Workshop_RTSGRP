\chapter{SWOT of Developmentprocess}
This chapter will look at the strengths, weaknesses, opportunities, and threats of using an agile development process as scrum.

\begin{description}
	\item [Strengths]\hfill\\
		\begin{itemize}
		\item Scrum allows to refit the requirements as new discoveries have been made. Since this project is in a new field to us as students, it is great to be flexible in the requirements as they are almost bound to change.
		\item Having a daily scrum meeting helps spreading the information in the group so everyone knows what is going on, what are the problems with it etc, which means that better discussions in the group will occur as more people are knowledgeable of the material.
		\item Having deadlines in the form of sprints can help reduce the risk of having a very very busy scheduele close to the final deadline.
		\item Agile teams work best if the developers are in the same room for increased interactions. We are as a group fitted with just such a room.
		\end{itemize}
	\item [Weaknesses]\hfill\\
		\begin{itemize}
		\item We do not have a customer which is what agile methods typically rely heavliy upon, but we have to act as such amongst ourselves, deciding on what is important with good arguments for our choices etc instead.
		\item A slippery slope is saying we are doing scrum, and not just ending up doing ad-hoc developing, where there is little sense of direction and planning.
		\end{itemize}
	\item [Opportunities]\hfill\\
		\begin{itemize}
		\item It gives us opportunites to work in iterations, which can be a very nice thing when working with comples problems, seperating the problems into smaller pieces and slowly building the entire project.
		\item Agile methods gives us the opportunity to try and use something in the project, and finding its limitations as we go and thus adapting.
		\end{itemize}
	\item [Threats]\hfill\\
		\begin{itemize}
		\item A lack of good documentation is a threat which is not often found in linear development.
		\item Programming the same code over and over again, because the phase of requirements engineering is not thorough enough, which can cause a lot of unnecessary work.
		\end{itemize}
\end{description}