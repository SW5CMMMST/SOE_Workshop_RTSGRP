\chapter{Implementation Plan} % (fold)
\label{cha:implementation_plan}
To implement the agile Scrum method in our project, we will be setting up a physical scrum board with columns for “Stories”, “Todo”, “Doing” and “Review”. Each row of the scrum board is to be used for the different stories in the current sprint. 

Some of the different roles needed to implement Scrum will be delegated as follows:
\begin{description}
    \item[The Scrum master] will be assigned to a new group member at the beginning of each sprint.
    \item[The product owner] is not directly present in this implementation of Scrum, but will be represented by the entire group in collaboration.
    \item[The development team] will likewise be consisting of the entire group.
\end{description}

Every sprint will be initiated during a sprint planning meeting, where the group decides which stories from the backlog should be worked on in the coming sprint. 
A typical sprint is estimated to last one week but longer sprints may be used if needed. 

Each day typically begins with a daily Scrum meeting, where all members of the development team answers three questions: \emph{1)} What have you done since last daily scrum; \emph{2)} What are you going to work on until next daily scrum; and \emph{3)} Is there anything obstructing that. 
In this implementation of Scrum there is room for skipping daily scrums, if the group decides that it will not benefit the current work flow e.g. if no group member has finished a task and no immediate obstacles are present. 

At the end of a sprint, the group conducts a sprint review meeting, where the sprint and work of the development team is reviewed. 
This meeting is also to be closely connected with a working release of the product, but because of the structure of the project, which requires a lot of documentation and analysis in the student report, the early iterations (sprints) will not result in working releases.

% chapter implementation_plan (end)

