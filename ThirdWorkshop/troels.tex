\section{Evaluation of practices}
With the SOE course we have taken a close look at the processes we have used for this semester, and the ones we have adapted from our use on previous semesters.
The most notable of these is Scrum, we have used a relaxed version of Scrum, as there is not a product owner available. 
The primary advantage we had using Scrum was the Scrum board, and the mindset of (bi-)weekly sprints.
This have kept our group in sync with each other, and have kept us on the right track. 
However we have moved so far from the original Scrum model, we would like to move back closer to it.
This includes being better to have proper sprint meetings, harder deadline, and a better backlog etc. 

A significant problem in the project group have been that the elective courses have taken up a lot of time, sometimes causing the group to not do proper stand-up meetings for 2 days in a row. 
This problem was in part solved by doing work as two sub-groups, however this was not a good solution in the long run, as there can be an ``us versus them'' mentality. 

Finally we shouldn't be nervous about making big changes, especially early on, to improve the quality of our work. 
This might mean that we remove something one or more group members have spent a lot of time on, this is also why it can be hard to remove. 
However it is often the correct choice as new information is gathered, and we as a group are smarter on the subject and therefore can make better decisions.
