\chapter{Evaluation} % (fold)
\label{cha:evaluation}
This chapter consists of evaluations of: the development process; the consequences of the tools techniques, and practices in use; and future practices and changes that should be applied.
\myref{tab:dev-process-eval} and \myref{tab:ttandp-eval} also presents the different evaluations.
% chapter evaluation (end)   

\section{Evaluation of the Development Process} % (fold)
\label{sec:evaluation_of_the_development_process}

The elements from our development process, and version of scrum, which provided the most significant improvements to the project, were the use of a scrum board and the daily scrum meetings.
However the incorporation of a scrum board was difficult at first, because of the amount of courses the project group had to attend.
The important updating of the scrum board was also somewhat neglected at times, meaning that because of bigger tasks and user stories, some group members ended up with ad-hoc tasks which was not listed on the scrum board.
This could have been solved by analysing the different stories better and writing more explicit tasks.
Moreover the daily scrum meetings proved impractical at times, when no significant work had been done, usually due to excessive course activity or the same problem as discussed before with unfulfilling tasks for different stories.
It should be noted that generally the scrum board and the daily scrums, helped the group stay synchronised with their work, and ensured that dependencies among tasks and stories could be quickly resolved.

One of the downsides in regards to the way scrum is implemented as our development process is the lack of a product owner.
This made it difficult to decide which features to develop and especially in what order, hereby inhibiting some dependencies in the project.
Additionally the role of the scrum master could benefit from being granted more focus, which in turn would make the workflow more regulated and ordered.
Finally the group work would be strengthened if harder and stricter deadlines were applied, since the amount of work done was very minor at times.   

Furthermore, since the project group had little knowledge of embedded systems at the beginning of the project, it was very helpful to not have to plan the whole project ahead of time, and instead take the agile approach with incremental problem solving.

% section evaluation_of_the_development_process (end)