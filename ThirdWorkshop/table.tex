\begin{sidewaystable}[]
\centering
\caption{Development Process}
\label{tab:dev-process-eval}
\begin{tabularx}{\textwidth}{|l|X|X|X|X|}
\hline
Scrum
& How do we do this activity?                                                                                                                                                                             & Does it work?                                                                                                                                                                                                                                                                              & How do we know?                                                                                                                                                                                                                                                                                                                                                                                                                                         & How can practice be improved?                                                                                                                                                                                                                                                                \\ \hline
Daily Scrum                 
& Stand-up meeting, yesterday, today, problems?                                                                                                                                                           
& Yes, when we do it, and people take it seriously. Sometimes off-tracking.                                                                                                                                                                                                                   
& In the periods where the technique is used properly, the productivity of the group has been higher, and the synchronization in the group is a lot better. We cannot show data for this, but this is the result we can feel.                                                                                                                                                                                                                             & More controlled process. A dedicated Scrum master instead of a interchangeable scrum-master, should result in us following the process better.                                                                                                                                                \\ \hline
Scrum Board                 
& Physical board in group room, using post-its.                                                                                                                                                           
& Yes, however sometimes we have been bad at keeping it updated. Mainly because of people having different course schedules.                                                                                                                                                                  
& Sometimes when we had a lot of courses, it would often be outdated. 
Some of the group had one course, while the rest had another, which resulted in us not being at the university at the same time, which had the results of us not updating the physical Scrum board. 
Later into the project when courses are not as intense anymore, we can see an improvement in the usage of the board, and also how it creates an image of progress for the group & Agreeing on a meeting time to actually have the daily Scrum to update the scrum board, even when courses interfere. 
Some days half the group would not show in the group room due to courses.                                                                                                \\ \hline
Scrum Master                
& Scrum-master will be passed around from each sprint.                                                                                                                                                     
& No, not really, but it might be because of the very faint usage of the Scrum master, which was basically just controlling the daily scrum meeting. 
And it was not always announced who was Scrum master, so someone each day would control the meeting, which might have caused confusion. & The meetings are not as controlled, and the scrum board has not been used to full effect. The backlog was not used properly either, however, this is not mentioned elsewhere in the table. A proper scrum master role might results in this being followed better.                                                                                                                                                                                      & Maybe some kind of explicit indication of who the scrum-master is, could help remove potential confusion, e.g. a badge or hat of some kind. Moreover a specific set of procedures could be made available to the scrum-master, which in turn would make the role of scrum-master more tangible. \\ \hline
Product Owner               & We have not had a dedicated PO, this is the result of this being a student project without an external stakeholder. The role of the product owner has been taken by all group members in collaboration. & Yes/No, sometimes it is easy to choose what tasks to do in the next sprint, however a dedicated PO would have been very helpful at times.                                                                                                                                                  & We have no data to show how it would have been different with a dedicated PO.                                                                                                                                                                                                                                                                                                                                                                           & Either getting a dedicated PO (hard for student projects) or assigning a group member to fulfill the role each sprint.                                                                                                                                                                       \\ \hline
\end{tabularx}
\end{sidewaystable}

\begin{sidewaystable}[]
\centering
\caption{Tools Techniques and Practices}
\label{tab:ttandp-eval}
\begin{tabularx}{\textwidth}{|l|X|X|X|X|}  
\hline
Activities       & How do we do this activity?                                                                                                                                                                                                 & Does it work?                                                                                                                                                                                                                                                   & How do we know?                                                                                                                                                                                                                                                                                                                                                                    & How can practice be improved?                                                                                                                                                                                                                                                                   \\ \hline
Pair-Programming & When a group member takes a task which involves programming, they ask if someone wants to pair program with them.                                                                                                           & Often it does, the code produced is more thought out, than that which a single person produces. However it uses two people instead of one. Though sometimes one of the people involved would be distracted and do something else, often not project related.    & The evaluation of the code finds fewer errors in general on code which was pair-programmed. Moreover when programming in pairs, obstacles appear to be solved significantly faster.                                                                                                                                                                                                & Be more consistent about pair-programming and not getting distracted. Either by turning off electronic devices, or by physically moving away from the (loud) group room.                                                                                                                        \\ \hline
Refactoring      & When some code or text, which is unclear or ``broke'', is encountered, it should be refactored if possible.                                                                                                                  & Most of the time, however the retrofitting refactoring is less used in the code base of the project because of pair programming.                                                                                                                                & It is hard to tell if refactoring works, since the code before the refactoring rarely is seen by anyone but the writer, with the exception of pair programming.                                                                                                                                                                                                                    & Everybody should be encouraged to be more aggressive in the refactoring of their own work, especially when it comes to source code. The aggressiveness should be enforced upon constructions that seems clear to the programmer but perhaps not to other developers.                            \\ \hline
Version Control  & We use Git as VC for our code, project report, UPPAAL-models and presentations. Each task is worked in a branch, after completion it is merged into the master branch.                                                      & Yes, it is for us unimaginable not to use a VCS. This allows us to seamlessly develop six people concurrently, do peer review and collect it all easily.                                                                                                        & We have frequently developed six people simultaneously, and haven't had any issues doing peer-review and collecting.                                                                                                                                                                                                                                                                & Using better names for branches and better commit messages. This can make it easier to identify what changes has been made by looking on Github or in your local repository.                                                                                                                   \\ \hline
Peer Review      & We peer review every time a task on the Scrum board is in the “Review” column, and nothing is done before it has been through this review process. This happens for all project parts, code, tests, text for the paper etc. & It works yes, many times mistakes have been corrected using this technique. Sometimes one group member might write a section in a way the peer reviewer does not agree with, and this sparks a discussion of the contents, which heightens the quality greatly. & We know because we can see the changes on git from when a pull request is created on a task. Here the peer reviewer will look over the pull request, spell check the files, and make corrections if errors or inconsistencies are found.These bigger changes always happen in discussion with the original author, which forces us to think of why we are writing what we are etc. & This activity might suffer from the sometimes non-optimal use of the scrum board, which means some tasks might have slipped through the reviewing process now and then, however refactoring might catch these errors, but we do believe everything still in the project has been peer reviewed. \\ \hline
\end{tabularx}
\end{sidewaystable}